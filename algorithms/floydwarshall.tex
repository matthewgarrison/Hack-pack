\startsection{Floyd-Warshall's algorithm}

\textit{Runtime: O(V\textsuperscript{3})}

This algorithm finds the shortest path between all indexes of a weighted or unweighted graph. Negative edge weights are allowed, but negative cycles are not (ie. the sum of the weights of the edges of the cycle is negative).

Scan the connections into a $n \times n$ \inline{adj} array, where 0 is no cost (ie. from a vertex to itself), \inline{INFINITY} is no edge, and anything else is the cost of the edge.

The path array is only used if you need to find the actual path from i to j. Once the algorithm has been run, the value at (i, j) is the index of the last vertex reached before j when travelling on the shortest path from i to j. For example, if the path from i to j is [i -> 5 -> 3 -> 2 -> j], the value at (i, j) will be 2. You construct the array with the previous vertex for each edge (ie. i) and use -1 to indicate no edge.

\lstinputlisting{algorithms/code/floydwarshall1.java}

The algorithm:

\lstinputlisting{algorithms/code/floydwarshall2.java}

To find the actual path from index i to index j using the path array, you need to work backwards. You can add them to an ArrayList, and then print it out backwards.

\lstinputlisting{algorithms/code/floydwarshall3.java}

\newpage