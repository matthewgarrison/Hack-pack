\startsection{Graphs}

General graph algorithms.

\subsection*{BFS and DFS}

\subsection*{DFS-Ordering}

The DFS-Ordering of a tree orders the vertexes such that all of a vertex's children are on \inline{[startIdx, endIdx]} (and the vertex itself is at \inline{startIdx}). This version (with initial \inline{idx=-1}) will produce indexes on \inline{[0, numVertexes-1]}. You can start with \inline{idx=0} to produce indexes on \inline{[1, numVertexes]}.

\lstinputlisting{algorithms/code/graphs_dfsordering1.java}

\subsection*{Eulerian paths/cycles}

Eulerian path: a path that visits every vertex exactly once

Eulerian cycle (or circuit): a Eulerian path that is a cycle (ie. starts and ends at the same vertex)

Semi-Eulerian graph: a graph containing a Eulerian path

Eulerian graph: a graph containing a Eulerian cycle

Undirected: 
\begin{itemize}
    \item Eulerian cycle: iff there are no vertexes of odd degree
    \item Eulerian path: iff there are zero or two vertexes of odd degree (if two, those are the starting and ending vertexes)
\end{itemize}

Directed:
\begin{itemize}
    \item Eulerian cycle: iff every vertex has equal in-degree and out-degree and all those vertexes belong to a single connected component
    \item Eulerian path: iff at most one vertex has $in - out = 1$, at most one vertex has $out - in = 1$, every other vertex has equal in-degree and out-degree, and all vertexes belong to a single component in the equivalent undirected graph
\end{itemize}

\textbf{Fluery's algorithm}

\textit{Runtime: O(E\textsuperscript{2})}

The algorithm is \textit{O(E)}, however it must run Tarjan's bridge-finding algorithm (which is itself \textit{O(E)} every iteration.

\begin{enumerate}
    \item Start at a vertex of odd degree, or if there isn't any, an arbitrary vertex
    \item Choose a non-bridge edge of the current vertex, or if there is no non-bridge edge, choose the remaining edge
    \item Move to the other endpoint of the edge and delete the edge
\end{enumerate}

At the end of the algorithm there are no edges left, and the sequence from which the edges were chosen forms an Eulerian cycle if the graph has no vertices of odd degree, or an Eulerian trail if there are exactly two vertices of odd degree.

\textbf{Hierholzer's algorithm}

\textit{Runtime: O(E)}

\begin{itemize}
    \item Choose any starting vertex v, and follow a trail of edges from that vertex until returning to v. It is not possible to get stuck at any vertex other than v, because the even degree of all vertices ensures that, when the trail enters another vertex w there must be an unused edge leaving w. The tour formed in this way is a closed tour, but may not cover all the vertices and edges of the initial graph.
    \item As long as there exists a vertex u that belongs to the current tour but that has adjacent edges not part of the tour, start another trail from u, following unused edges until returning to u, and join the tour formed in this way to the previous tour.
\end{itemize}

By using a data structure such as a doubly linked list to maintain the set of unused edges incident to each vertex, to maintain the list of vertices on the current tour that have unused edges, and to maintain the tour itself, the individual operations of the algorithm (finding unused edges exiting each vertex, finding a new starting vertex for a tour, and connecting two tours that share a vertex) may be performed in constant time each

\subsection*{Hamiltonian paths/cycles} 

Hamiltonian path: a path that visits every vertex exactly once

Hamiltonian cycle: a Hamiltonian path that is a cycle (ie. starts and ends at the same vertex)

Hamiltonian graph: a graph containing a Hamiltonian cycle \\
\indent Both complete graphs (with more than 2 vertexes) and cycle graphs are Hamiltonian graphs. \\
\indent All Hamiltonian graphs are biconnected, but not all biconnected graphs are Hamiltonian.

Determining if a graph contains a Hamiltonian path or cycle is NP-complete.

\subsection*{Misc terminology}

Articulation point (AKA cut vertex): removing this vertex disconnects the graph

Biconnected: removing any single vertex does not disconnect the graph (ie. it doesn't contain any articulation points)

Bridge: removing this edge disconnects the graph

Cycle graph (C\subscript{n}): the graph with n vertexes containing a single cycle (AKA just a circle). C\subscript{0}, C\subscript{1}, and C\subscript{2} are not defined.

Complete graph (K\subscript{n}): the graph with n vertexes and every possible edge (not a multigraph). K\subscript{n} has $\binom{n}{2}$ or $\frac{n(n-1)}{2}$ edges.

Empty graph ( $\overline{K_n}$ ): the graph with n vertexes and no edges

k-connected: removing any k vertexes does not disconnect the graph

\subsection*{Toposort}

Khan's algorithm: 
\begin{itemize}
    \item Take all the vertexes with indegree 0 and put them into a queue.
    \item Remove a vertex and subtract the indegree of all its neighbors by 1. If any of those neighbors now have indegree 0, add them to the queue.
    \item Repeat until the queue is empty.
\end{itemize}

You can use a PriorityQueue (sorting on index) to get the lexographically first toposort.

\textbf{Checking if a TopoSort is valid}

\lstinputlisting{algorithms/code/graphs_topo1.java}

\subsection*{Trees}

Diameter: longest path in the tree. Procedure to find it: run a BFS starting at an arbitrary node. Then, run a second BFS starting at the last node you visited in the first BFS. The largest distance found in the second BFS is the diameter. This finds the diameter in terms of edges - to find it in terms of nodes, add 1.

Center: the node that minimizes remoteness from all other nodes (ie. it minimizes the maximum distance to any node). If the diameter is odd (when edge-counting; if node-counting, when it's even), there will be two centers. The center will be the node(s) with distance \inline{diameter/2} after the second BFS.

Centroid: if you remove this node, the maximum size components remaining are minimized

\subsection*{Two-coloring}

\textit{Runtime: O(n)}

Works in Bipartite graphs.

\lstinputlisting{algorithms/code/graphs_twocoloring1.java}


\newpage