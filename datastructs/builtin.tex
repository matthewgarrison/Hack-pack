\startsection{Built-in data structures}

\textbf{Maps}

A class that implements the Map interface allows you to map Keys to Values. Each Key can map to at most one Value (no duplicate Keys), but duplicate Values are possible. 

A \inline{TreeMap} sorts the Keys according to their natural ordering or a provided Comparator. This allows you to do things like \inline{pollFirstKey()} or \inline{lowerKey(Key k)} (returns the greatest Key less than k).

A \inline{LinkedHashMap} orders the Keys according to the order in which they were inserted. “Note that insertion order is not affected if a key is re-inserted into the map. (A key k is reinserted into a map m if \inline{m.put(k, v)} is invoked when \inline{m.containsKey(k)} would return true immediately prior to the invocation.)” 

A \inline{HashMap} orders the Keys randomly, but is faster than the other two.

Iterates over the values in the map, where K and V are the types of the Keys and Values in the map:

\lstinputlisting{datastructs/code/builtin1.java}

If you wish to use a custom class as Keys, then the class needs to implement \inline{equals()} and \inline{hashCode()}.

\textbf{Queues and stacks}

Use \inline{ArrayDeque}, as it’s faster than a \inline{Stack} or \inline{LinkedList}.

Queues: elements are added to the end and removed from the front. Use \inline{addLast()}, \inline{peekFirst()}, and \inline{pollFirst()}.

Stack: elements are added to and removed from the front. Use \inline{addFirst()}, \inline{peekFirst()}, and \\ \inline{pollFirst()}.

\textbf{Sets}

\inline{TreeSet} and \inline{LinkedHashSet} have the same ordering as \inline{TreeMap} and \inline{LinkedHashMap}.

To iterate over the elements of a Set, you can use the Iterator returned by \inline{set.iterator()} or with a for-each loop.

\newpage