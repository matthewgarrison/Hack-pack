\startsection{Binary and ternary search}

\subsection*{Binary search}

\textit{Runtime: O(log(n))}

A binary search will search through an increasing or decreasing function or \underline{sorted} array for a desired value.

\lstinputlisting{general/code/binternsearch1.java}

If the array contains multiple occurrences of the value, this will find the first one. To find the last occurrence, change line \#6 to be lo = mid + 1.

\lstinputlisting{general/code/binternsearch2.java}

Binary search for the first \inline{true} value. The answer is at \inline{lo}. To search for the last \inline{true}, negate the if-statement, and the answer will be at \inline{lo-1}. 

\lstinputlisting{general/code/binternsearch3.java}

You can also do a binary search on floating point numbers. Basically, you go for repetition, instead of checking if lo exceeds hi. This is most often used when binary searching a function. The number of repetitions you need varies depending on the problem, but 100 is often enough. Make sure to stay within the time limit.

\lstinputlisting{general/code/binternsearch4.java}

Alternatively (answer at \inline{lo}):

\lstinputlisting{general/code/binternsearch5.java}

\subsection*{Ternary search}

A ternary search will find the lowest or highest value in a parabolic function. Like when using a binary search on floating points numbers, we go for repetition. Answer is at \inline{low}.

\lstinputlisting{general/code/binternsearch6.java}

Ternary search on integers:

\lstinputlisting{general/code/binternsearch7.java}

\newpage
