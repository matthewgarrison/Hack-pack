\startsection{Miscellaneous}

\subsection*{Anonymous classes}

Anonymous classes are classes that are declared inline. For example, this is a PriorityQueue with a Comparator that was declared inline.

\lstinputlisting[style=display]{general/code/misc_anon1.java}

You can also use lambdas with anonymous classes.

\lstinputlisting[style=display]{general/code/misc_anon2.java}

\subsection*{Checking equality in doubles}

It can be hard to check equality in doubles due to floating point errors. This method will check for equality if floating point errors are possible:

\lstinputlisting[style=display]{general/code/misc_doubleequal.java}

10\textsuperscript{-9} is generally a good value for \lstinline[style=inline]{EPSILON}. Also, most problems involving doubles will tell you how much error two values can have and still be considered equal.

\subsection*{Evaluating polynomials at a point}

Given an array of the coefficients of a polynomial you can find the value of the polynomial evaluated at a number num (ie. P(num), where P(x) is the polynomial). This works by the remainder theorem.

Note that the array of coefficients should be in descending order. If \lstinline[style=display]{P(x) = 3x$^2$ – 2x + 1}, then the array should equal \lstinline[style=display]{\{3, -2, 1\}}.

\subsection*{Useful classes}

ArrayDeque: stacks and queues

BitSet: useful for problems with large bitmasks (max size is $^{2Integer.MAX\_VALUE}$)

Formatter: the JavaDoc lists printf syntax

Line2D: see if lines intersect

LocalDate, LocalTime, LocalDateAndTime: time problems

MathContext and RoundingMode

Pattern: the JavaDoc lists regex syntax

Rectangle2D