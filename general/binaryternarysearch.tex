\startsection{Binary and ternary search}

\subsection*{Binary search}

\textit{Runtime: O(log(n))}

A binary search will search through an increasing or decreasing function or \textbf{sorted} array for a desired value.

\lstinputlisting[style=display]{general/code/binternsearch1.java}

If the array contains multiple occurrences of the value, this will find the first one. To find the last occurrence, change line \#7 to be lo = mid + 1.

\lstinputlisting[style=display]{general/code/binternsearch2.java}

You can also do a binary search on floating point numbers. Basically, you go for repetition, instead of checking if lo exceeds hi. This is most often used when binary searching a function. The number of repetitions you need varies depending on the problem, but 100 is often enough. Make sure to stay within the time limit.

\lstinputlisting[style=display]{general/code/binternsearch3.java}

Alternatively:

\lstinputlisting[style=display]{general/code/binternsearch4.java}

\subsection*{Ternary search}

A ternary search will find the lowest or highest value in a parabolic function. Like when using a binary search on floating points numbers, we go for repetition.

\lstinputlisting[style=display]{general/code/binternsearch5.java}

\newpage
