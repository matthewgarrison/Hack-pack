\startsection{Infix, prefix, postfix}

\subsection*{Infix to postfix (Shunting-yard algorithm)}

This implementation assume each number and operator is separated by spaces.

\lstinputlisting{general/code/infix1.java}

\subsection*{Postfix to infix}

Converts postfix to infix, with only the necessary parenthesis. If you want a fully parenthesized equation, you can just use Strings and \inline{String combo = "( "+two+" "+token+" "+one+" )";}

\lstinputlisting{general/code/infix2.java}

\subsection*{Prefix}

Infix to prefix: reverse the expression, replace every \inline{'('} with \inline{')'} and vice-versa, run Shunting-Yard, then reverse the result.

Prefix to infix: reverse the expression, run the postfix to infix code, reverse the result, and then replace every \inline{'('} with \inline{')'} and vice-versa.

\subsection*{Evaluation}

Calculates the value of a postfix expression, which has spaces between numbers and operators.

\lstinputlisting{general/code/infix3.java}

Calculates the value of an infix expression.

\lstinputlisting{general/code/infix4.java}

\newpage