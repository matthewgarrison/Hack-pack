\startsection{Bits}

\textbf{NOT}: Because Java uses two's complement, \inline{\textasciitilde n} returns \inline{abs(n) - 1}

\textbf{Left shift}: left shift of x by y is equivalent to multiplying x by 2\textsuperscript{y}

\textbf{Right shift}: right shift of x by y is equivalent to dividing x by 2\textsuperscript{y}

\textbf{Modding by a power of two}: If y = 2\textsuperscript{n}, then \inline{x \% y} is equivalent to \inline{x \& (y - 1)}

\textbf{Check if \inline{n} is a power of two}: \inline{n != 0 \&\& (n \& (n - 1)) == 0}

Alternatively: \inline{Integer.bitCount(n) == 1}

\textbf{Check if \inline{x} and \inline{y} have opposite signs}: \inline{(x $\wedge$ y) < 0}

\textbf{Determine powers of 2}: \inline{Integer.highestOneBit(n)} returns the largest power of 2 less than or equal to $n$ (ie. a number with a single one bit in the same position as the highest one bit in $n$). For example, \inline{Integer.highestOneBit(10) == 8} and \inline{Integer.highestOneBit(16) == 16}.

Next power of two: \inline{Integer.highestOneBit(n) << 1;}

\inline{Integer.numberOfTrailingZeros(Integer.highestOneBit(n))} will return the position of that one bit. This is also equivalent to $floor(log_2 (n))$

Lowest one bit (there's also an \inline{Integer} method): \inline{n \& (-n)}

\textbf{Undoing XOR}: if \inline{c = a $\wedge$ b}, then \inline{a = c $\wedge$ b} and \inline{b = c $\wedge$ a}

In other words, \inline{a = (a $\wedge$ b) $\wedge$ b}

\textbf{Toggle bitmasks}: If you have a bitmask and you want to toggle the n\textsuperscript{th} bit, \inline{mask $\wedge$= 1 << n;}

\textbf{Perfect square check}:

\lstinputlisting{general/code/bits1.java}




\newpage