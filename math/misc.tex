\startsection{Miscellaneous}

\subsection*{Catalan numbers}

The only Catalan numbers $C_n$ that are odd are those for which $n=2^k-1$. All others are even.

The only prime Catalan numbers are $C_2=2$ and $C_3=5$.

$C = \{1, 1, 2, 5, 14, 42, 132, 429, 1430, 4862, 16796, 58786, 208012, 742900, 2674440, 9694845, 35357670\}$

$$
    C_n = \frac{1}{n+1}\binom{2n}{n} = \binom{2n}{n} - \binom{2n}{n+1} = \frac{(2n)!}{(n+1)!n!} = \prod_{k=2}^{n} \frac{n+k}{k} = \frac{1}{n+1}\sum_{i=0}^{n}\binom{n}{i}^2 = \int_{0}^{4}x^n\frac{1}{2\pi}\sqrt{\frac{4-x}{x}} dx
$$

$$C_0 = 1$$
$$C_{n+1} = \sum_{i=0}^{n}C_i C_{n-i} = \frac{2(2n + 1)}{n+2}C_n$$


\subsection*{Checking equality in doubles}

It can be hard to check equality in doubles due to floating point errors. This method will check for equality if floating point errors are possible:

\lstinputlisting{math/code/misc_doubleequal1.java}

10\textsuperscript{-9} is generally a good value for \inline{EPSILON}. Also, most problems involving doubles will tell you how much error two values can have and still be considered equal.

\subsection*{Coupon collector's problem}

Suppose there is an urn with n different coupons. What is the probability that more than t sample trials are needed to collect all n coupons? Or, given n coupons, how many coupons do you expect you need to draw with replacement before having drawn each coupon at least once?

Alternatively, how many times do you expect to need to roll an n-sided die in order to have every number come up at least once?

The answer: n * H\subscript{n}, where n is the number of coupons and H\subscript{n} is the n\textsuperscript{th} Harmonic number.

$$H_n = 1 + \frac{1}{2} + \frac{1}{3} + ... + \frac{1}{n} = \sum_{k=1}^n \frac{1}{k}$$

\subsection*{Finding the average of two values}

Obviously, you can do it like this: \inline{int c = (a + b) >> 1;}

However, that can lead to overflows. This will not: \inline{int c = a + ((b - a) >> 1);}

Alternatively: \inline{((x $\wedge$ y) >> 1) + (x \& y)}

\subsection*{Evaluating polynomials at a point}

Given an array of the coefficients of a polynomial you can find the value of the polynomial evaluated at \inline{x} (ie. $f(x)$). This works by the remainder theorem.

\lstinputlisting{math/code/misc_evalpoly1.java}

Note that the array of coefficients should be in descending order. If \inline{P(x) = 3x\textsuperscript{2} - 2x + 1}, then the array should equal \inline{\{3, -2, 1\}}.

\subsection*{Log properties}

$$log_b(x) = \frac{log_y(x)}{log_y(b)} \text{, where } y \text{ is any base}$$

\subsection*{Primality}

This is a longer isPrime method, but one that is faster than the “na\"{\i}ve version.

\lstinputlisting{math/code/misc_primality1.java}

\newpage