\startsection{Euler's totient function}

$\phi (n)$, or $\varphi (n)$, is the number of integers $k$ in the range $1 \leq k \leq n$ that are coprime to $n$. Properties:
\begin{itemize}
    \item ${\displaystyle \phi (n) = n \prod_{p \divides n} \Big(1 - \frac{1}{p} \Big)}$, where $p$ is the distinct prime factors of $n$
    \item If $p$ is prime, $\phi (p) = p-1$
    \item $n^e \mymod{m} = n^{e \mymod{\phi(m)}} \mymod{m}$
    \item $\phi (nm) = \phi (n) \phi (m) \frac{d}{\phi (d)}$, where $d = gcd(n, m)$
    
    Special cases:
    \begin{itemize}
        \item If $gcd(n, m) = 1$, then $\phi (nm) = \phi (n) \phi (m)$
        \item 
        $\phi (2m) = 
        \begin{cases}
            2 \phi (m) &\quad\text{if } m \text{ is even} \\
            \phi (m) &\quad\text{if } m \text{ is odd}
        \end{cases}$
        \item $\phi (n^m) = n^{m-1} \phi(n)$
    \end{itemize}
    \item Euler's theorem: if $a$ and $n$ are coprime, $a^{\phi (n)} \equiv 1 \mymod{n}$
    
    \begin{itemize}
        \item The special case where $n$ is prime is Fermat's Little Theorem: $a^p \equiv a \mymod{p}$
    \end{itemize}
    \item If $p$ is prime and $k \geq 1$, then
    $\phi (p^k) = p^k - p^{k-1} = p^{k-1} (p - 1) = p^k \big( 1 - \frac{1}{p} \big)$
    \item If ${\displaystyle n = \prod p^k }$ (ie. the prime factorization of $n$), then ${\displaystyle \phi (n) = \prod \phi \big( p^k \big) }$
    \item Divisor sum:
    ${\displaystyle \sum_{d \divides n} \phi(d) = n}$
    \item $a \divides b \implies \phi(a) \divides \phi(b)$
    \item $n \divides \phi(a^n - 1)$, for $a, n > 1$
    \item $\phi (lcm(n, m)) \phi( gcd(n, m)) = \phi (n) \phi (m)$
    \item $\phi (n)$ is even for $n \geq 3$. Moreover, if $n$ has $r$ distinct odd prime factors, $2^r \divides \phi(n)$.
    \item For any $a > 1$ and $n > 6$ such that 4 doesn't divide $n$, there exists an $l \geq 2n$ such that $l \divides \phi (a^n - 1)$
    \item $\frac{\phi (n)}{n} = \frac{\phi (rad(n))}{rad(n)}$, where $rad(n)$ is the radical of $n$ and ${\displaystyle rad(n) = \prod_{p \divides n} p}$
    \item Menon's identity: ${\displaystyle \sum_{1 \leq k < n} gcd(k-1, n) = \phi (n) d(n)}$, where $d(n)$ is the number of divisors of $n$ and $gcd(k,n) = 1$
\end{itemize}

There's a sieve in the Sieves document, but if you want to directly calculate $\phi (n)$ (2\textsuperscript{nd} version is faster):

\lstinputlisting{math/code/euler1.java}

\lstinputlisting{math/code/euler2.java}

\newpage