\startsection{Matrices}

A $n \times m$ matrix is a matrix with $n$ rows and $m$ columns. A square matrix is a $n \times n$ matrix. The identity matrix is a $n \times n$ matrix where the elements on the main diagonal (top-left to bottom-right) are ones and the rest are zeroes.

\subsection*{Addition and subtraction}

Add each number to its pair in the second matrix. NOTE: the matrices must be the same size.

\begin{equation*}
    \begin{bmatrix}
        2 & 6 & 12 \\
        7 & -4 & 10
    \end{bmatrix}
    +
    \begin{bmatrix}
        15 & -44 & -3 \\
        8 & 0 & 9
    \end{bmatrix}
    =
    \begin{bmatrix}
        17 & -38 & 9 \\
        15 & -4 & 19
    \end{bmatrix}
\end{equation*}

\subsection*{Scalar multiplication}

Multiply every number in the matrix by the scale factor.

\begin{equation*}
    2 * 
    \begin{bmatrix}
        4 & 0 \\
        1 & -9
    \end{bmatrix}
    =
    \begin{bmatrix}
        8 & 0 \\
        2 & -18
    \end{bmatrix}
\end{equation*}

\subsection*{Multiplication (dot product)}

To get the number in row i and column j of the resulting matrix, we multiply each number in row i of the first matrix by each number in column j of the second and add the products.

If C = AB, for an $n \times m$ matrix A and an $m \times p$ matrix B, then C is an $n \times p$ matrix. Note that the number of columns in matrix A and the number of rows in matrix B must be the same.

\lstinputlisting{math/code/matrices1.java}

\subsection*{Exponentiation}

NOTE: Raising a matrix to the zeroth power returns the identity matrix. The matrix must be square.

\lstinputlisting{math/code/matrices2.java}

\newpage