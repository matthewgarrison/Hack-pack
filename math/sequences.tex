\startsection{Sequences}

\subsection*{Bell numbers} 

The number of partitions a set of n elements has. (Partition of a set: grouping of the set's elements into non-empty subsets, in such a way that every element is included in one and only one of the subsets.)

$B = \{1, 1, 2, 5, 15, 52, 203, 877, 4140, 21147, 115975, 678570, 4213597, 27644437, 190899322, 1382958545\}$

$$B_{n+1} = \sum^n_{k=0} \binom{n}{k} B_k \formsep B_n = \sum^n_{k=0} \genfrac{\{}{\}}{0pt}{}{n}{k}$$

The Bell numbers can be calculated with the Bell triangle (similar to Pascal's triangle), where B\subscript{n} is the first number in the n\textsuperscript{th} row.

\begin{center}
    \begin{tabular}{l}
    1\\
    1 2\\
    2 3 5\\
    5 7 10 15\\
    15 20 27 37 52\\
    \end{tabular}
\end{center}

\lstinputlisting{math/code/sequences_bell1.java}

\subsection*{Catalan numbers}

The only Catalan numbers $C_n$ that are odd are those for which $n=2^k-1$. All others are even.

The only prime Catalan numbers are $C_2=2$ and $C_3=5$.

$C = \{1, 1, 2, 5, 14, 42, 132, 429, 1430, 4862, 16796, 58786, 208012, 742900, 2674440, 9694845, 35357670\}$

$$
    C_n = \frac{1}{n+1}\binom{2n}{n} = \binom{2n}{n} - \binom{2n}{n+1} = \frac{(2n)!}{(n+1)!n!} = \prod_{k=2}^{n} \frac{n+k}{k} = \frac{1}{n+1}\sum_{i=0}^{n}\binom{n}{i}^2 = \int_{0}^{4}x^n\frac{1}{2\pi}\sqrt{\frac{4-x}{x}} dx
$$
$$C_0 = 1 \formsep C_{n+1} = \sum_{i=0}^{n}C_i C_{n-i} = \frac{2(2n + 1)}{n+2}C_n$$

\subsection*{Fibonacci sequence}

$F = \{0, 1, 1, 2, 3, 5, 8, 13, 21, 34, 55, 89, 144, 233, 377, 610\}$

$x$ is a part of the Fibonacci sequence iff one or both of $5x^2 + 4$ or $5x^2 - 4$ is a perfect square.

Every 3\textsuperscript{rd} number of the sequence is even and more generally, every k\textsuperscript{th} number of the sequence is a multiple of F\subscript{k}. In fact, it satisfies the stronger divisibility property $GCD(F_m, F_n) = F_{GCD(m, n)}$.

Any three consecutive Fibonacci numbers are pairwise coprime, ie. $GCD(F_n, F_{n+1}) = GCD(F_n, F_{n+2}) = GCD(F_{n+1}, F_{n+2}) = 1$.

Every prime number $p$ divides a Fibonacci number that can be determined by the value of $p \mymod{5}$. If p is congruent to 1 or 4 (mod 5), then $p$ divides F\subscript{p-1}, and if p is congruent to 2 or 3 (mod 5), then, p divides F\subscript{p+1}. The remaining case is that $p = 5$, and in this case $p$ divides F\subscript{p}.

You can use matrices to calculate the Fibonacci sequence. This matrix representation gives the following closed expression for the Fibonacci numbers:

\begin{equation*}
    \begin{bmatrix}
        F_{n+1} & F_n \\
        F_n & F_{n-1}
    \end{bmatrix}
    =
    \begin{bmatrix}
        1 & 1 \\
        1 & 0
    \end{bmatrix}
    ^n
    \formsep
    \begin{bmatrix}
        F_n \\
        F_{n-1}
    \end{bmatrix}
    =
    \begin{bmatrix}
        1 & 1 \\
        1 & 0
    \end{bmatrix}
    \begin{bmatrix}
        F_{n-1} \\
        F_{n-2}
    \end{bmatrix}
\end{equation*}

\textbf{Generalization}

If $F_0 = a$ and $F_1 = b$, then:

\begin{equation*}
    \begin{bmatrix}
        F_n \\
        F_{n-1}
    \end{bmatrix}
    =
    \begin{bmatrix}
        1 & 1 \\
        1 & 0
    \end{bmatrix}
    ^{n-1}
    \begin{bmatrix}
        b \\
        a
    \end{bmatrix}
\end{equation*}

If $a = b = 1$, then the Fibonacci sequence will be "shifted" to the left by one. For example, 8 is the 6\textsuperscript{th} Fibonacci number, but would be the 5\textsuperscript{th} in the generalized Fibonacci sequence.

(If $a = 0$ and $b = 1$, then it will be the normal sequence.)

\subsection*{Fourth-dimensional pyramidal numbers}

$a = {0, 0, 1, 6, 20, 50, 105, 196, 336, 540, 825, 1210, 1716, 2366, 3185, 4200, 5440, 6936, 8721, 10830, 13300, 16170}$
$$a(n) = \frac{n^2(n^2 - 1)}{12}$$

\subsection*{General sequences}

Sum of an arithmetic sequence: $\frac{n(a_1 + a_n)}{2}$

Sum of a geometric sequence: $a_1 ( \frac{1 - r^n}{1 - r} )$

Sum of infinite geometric sequence (where $|r| < 1$): $\frac{a}{1 - r}$

Sum of the first $n$ numbers: $\frac{n(n+1)}{2}$

Sum of the first $n$ odd numbers: $n^2$

Sum of the first $n$ even numbers: $n(n+1)$

Sum of the first $n$ squares: $\frac{n (2n+1) (n+1)}{6}$

Sum of the first $n$ cubes: $( \frac{n(n+1)}{2}) ^2$

\subsection*{Stirling numbers (first kind)}

The number of of permutations of $n$ elements with $k$ disjoint cycles (denoted $c(n, k)$, $\lvert s(n, k) \rvert$, or $\genfrac{[}{]}{0pt}{}{n}{k}$). For example, there are 6 permutations of 3 elements. Of those permutations, one has three cycles, three have two cycles, and two have one cycles. Therefore, $\genfrac{[}{]}{0pt}{}{3}{3} = 1$, $\genfrac{[}{]}{0pt}{}{3}{2} = 3$, and $\genfrac{[}{]}{0pt}{}{3}{1} = 2$.
$$\genfrac{[}{]}{0pt}{}{n+1}{k} = n \genfrac{[}{]}{0pt}{}{n}{k} + \genfrac{[}{]}{0pt}{}{n}{k-1} \text{, for } k > 0 \text{, with the initial conditions } \genfrac{[}{]}{0pt}{}{0}{0} = 1 \text{ and } \genfrac{[}{]}{0pt}{}{n}{0} = \genfrac{[}{]}{0pt}{}{0}{n} = 0 \text{, for } n > 0$$
$$\genfrac{[}{]}{0pt}{}{n}{k} = 0 \text{, if } k > n \formsep \genfrac{[}{]}{0pt}{}{n}{1} = (n-1)! \formsep \genfrac{[}{]}{0pt}{}{n}{n} = 1 \formsep \genfrac{[}{]}{0pt}{}{n}{n-1} = \binom{n}{2}$$
$$\genfrac{[}{]}{0pt}{}{n}{n-2} = \frac{1}{4} (3n-1) \binom{n}{3} \formsep \genfrac{[}{]}{0pt}{}{n}{n-3} = \binom{n}{2} \binom{n}{4} \formsep \sum_{k=0}^n \genfrac{[}{]}{0pt}{}{n}{k} = \genfrac{[}{]}{0pt}{}{n+1}{1}$$

\begin{center}
    \begin{tabular}{|c|c|c|c|c|c|c|c|c|c|c|c|}
        \hline
        n / k & 0 & 1 & 2 & 3 & 4 & 5 & 6 & 7 & 8 & 9 \\
        \hline
        0 & 1 & & & & & & & & & \\
        \hline
        1 & 0 & 1 & & & & & & & & \\
        \hline
        2 & 0 & 1 & 1 & & & & & & & \\
        \hline
        3 & 0 & 2 & 3 & 1 & & & & & & \\
        \hline
        4 & 0 & 6 & 11 & 6 & 1 & & & & & \\
        \hline
        5 & 0 & 24 & 50 & 35 & 10 & 1 & & & & \\
        \hline
        6 & 0 & 120 & 274 & 225 & 85 & 15 & 1 & & & \\
        \hline
        7 & 0 & 720 & 1764 & 1624 & 735 & 175 & 21 & 1 & & \\
        \hline
        8 & 0 & 5040 & 13068 & 13132 & 6769 & 1960 & 322 & 28 & 1 & \\
        \hline
        9 & 0 & 40320 & 109584 & 118124 & 67284 & 22449 & 4536 & 546 & 36 & 1 \\
        \hline
    \end{tabular}
\end{center}

\subsection*{Stirling numbers (second kind)}

The number of ways to partition a set of $n$ distinguishable objects into $k$ non-empty subsets (denoted $S(n, k)$ or $\genfrac{\{}{\}}{0pt}{}{n}{k}$).
$$S(n, k) = S(n-1, k-1) + k*S(n-1, k) = \frac{1}{k!} \sum_{j=0}^k (-1)^{k-j} \binom{k}{j} j^n$$

\begin{center}
    \begin{tabular}{|c|c|c|c|c|c|c|c|c|c|c|c|}
        \hline
        n / k & 0 & 1 & 2 & 3 & 4 & 5 & 6 & 7 & 8 & 9 & 10 \\
        \hline
        0 & 1 & & & & & & & & & & \\
        \hline
        1 & 0 & 1 & & & & & & & & & \\
        \hline
        2 & 0 & 1 & 1 & & & & & & & & \\
        \hline
        3 & 0 & 1 & 3 & 1 & & & & & & & \\
        \hline
        4 & 0 & 1 & 7 & 6 & 1 & & & & & & \\
        \hline
        5 & 0 & 1 & 15 & 25 & 10 & 1 & & & & & \\
        \hline
        6 & 0 & 1 & 31 & 90 & 65 & 15 & 1 & & & & \\
        \hline
        7 & 0 & 1 & 63 & 301 & 350 & 140 & 21 & 1 & & & \\
        \hline
        8 & 0 & 1 & 127 & 966 & 1701 & 1050 & 266 & 28 & 1 & & \\
        \hline
        9 & 0 & 1 & 255 & 3025 & 7770 & 6951 & 2946 & 462 & 36 & 1 & \\
        \hline
        10 & 0 & 1 & 511 & 9330 & 34105 & 42525 & 22827 & 5880 & 750 & 45 & 1 \\
        \hline
    \end{tabular}
\end{center}

The sum over the values for k of the Stirling numbers of the second kind, gives us $\displaystyle B_n = \sum_{k=0}^n \genfrac{\{}{\}}{0pt}{}{n}{k}$, ie. the n\textsuperscript{th} Bell number.


\newpage