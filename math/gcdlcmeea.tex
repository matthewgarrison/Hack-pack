\startsection{GCD, LCM, and EEA}

\textbf{Greatest Common Divisor}

\lstinputlisting{math/code/gcdlcmeea1.java}

\lstinputlisting{math/code/gcdlcmeea2.java}

\textbf{Least Common Multiple}

$$LCM(a, b) = \frac{ab}{GCD(a, b)}$$

\subsection*{Extended Euclidean Algorithm}

\textbf{Bezout's identity}: $ax + by = d$, where $a$ and $b$ are non-zero integers, $d$ is their GCD, and $x$ and $y$ are Bezout's coefficients.

\lstinputlisting{math/code/gcdlcmeea3.java}

The "quotients by the GCD" are a and b divided by the GCD. They may have an incorrect sign. Similarly, if either a or b is zero and the other is negative, the greatest common divisor that is output is negative, and all the signs of the output must be changed.

\subsection*{Misc}

A set of integers S = \{$a_1, a_2, .... a_n$\} can also be called coprime or setwise coprime if the greatest common divisor of all the elements of the set is 1.

If every pair in a set of integers is coprime, then the set is said to be pairwise coprime (or pairwise relatively prime, mutually coprime, or mutually relatively prime). Pairwise coprimality is a stronger condition than setwise coprimality; every pairwise coprime finite set is also setwise coprime, but the reverse is not true. For example, the integers 4, 5, 6 are setwise coprime, but they are not pairwise coprime (because gcd(4, 6) = 2).

Problem: Given the equation $ax + by = c$ and the values of a, b, and c, determine if there exists a solution for x and y. \\
\indent Solution: x and y exist if c is divisible by the GCD of a and b, because it means that a, b, and c are multiples of the same number.

Problem: Given the equation $ax - by = 0$, find x and y. \\
\indent Solution: x = b / GCD and y = a / GCD. The GCD is the largest common factor they share, so A * (B/gcd) = B * (A/gcd).

\newpage