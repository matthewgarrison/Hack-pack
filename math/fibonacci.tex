\startsection{Fibonacci sequence}

$F = \{0, 1, 1, 2, 3, 5, 8, 13, 21, 34, 55, 89, 144, 233, 377, 610\}$

If $5x^2 + 4$ or $5x^2 - 4$ is a perfect square, then x is part of the Fibonacci sequence.

You can use matrices to calculate the Fibonacci sequence. This matrix representation gives the following closed expression for the Fibonacci numbers:

\begin{equation*}
    \begin{bmatrix}
        F_{n+1} & F_n \\
        F_n & F_{n-1}
    \end{bmatrix}
    =
    \begin{bmatrix}
        1 & 1 \\
        1 & 0
    \end{bmatrix}
    ^n
    \formsep
    \begin{bmatrix}
        F_n \\
        F_{n-1}
    \end{bmatrix}
    =
    \begin{bmatrix}
        1 & 1 \\
        1 & 0
    \end{bmatrix}
    \begin{bmatrix}
        F_{n-1} \\
        F_{n-2}
    \end{bmatrix}
\end{equation*}

\lstinputlisting{math/code/fibonacci1.java}

\subsection*{Generalization}

If $F_0 = a$ and $F_1 = b$, then:

\begin{equation*}
    \begin{bmatrix}
        F_n \\
        F_{n-1}
    \end{bmatrix}
    =
    \begin{bmatrix}
        1 & 1 \\
        1 & 0
    \end{bmatrix}
    ^{n-1}
    \begin{bmatrix}
        b \\
        a
    \end{bmatrix}
\end{equation*}

If $a = b = 1$, then the Fibonacci sequence will be "shifted" to the left by one. For example, 8 is the 6\textsuperscript{th} Fibonacci number, but would be the 5\textsuperscript{th} in the generalized Fibonacci sequence.

(If $a = 0$ and $b = 1$, then it will be the normal sequence.)

\newpage