\startsection{Sieves}

\subsection*{Sieve of Eratosthenes}

\textit{Runtime: O(n log(log(n)))}

The Sieve of Erastosthenes computes primes by “marking off” multiples of all the primes before the current prime. The limit of how large a number you can check with this is based on how large an array you can make. However, you can “double” the max number by using a Sieve that only checks odd numbers. If you need a really large sieve, use a BitSet.

\lstinputlisting{math/code/sieves1.java}

You can count the number of prime factors a number has by changing lines 5-7 to:

\lstinputlisting{math/code/sieves2.java}

(A number i is prime if \inline{numPrimeFactors[i]} is 0.)

\subsection*{Totient sieve}

\lstinputlisting{math/code/sieves3.java}

\subsection*{Mod inverse sieve}

The modular inverse of \inline{i} \textit{mod} \inline{m}.

\lstinputlisting{math/code/sieves4.java}

\newpage